\section{Giới thiệu}
\begin{frame}{Giới thiệu}
\putlogo
\begin{itemize}
	\item \textbf{Bệnh án:} Tài liệu ghi chép về lịch sử thông tin sức khỏe và các lần điều trị của bệnh nhân 
	\begin{itemize}
		\item \textit{Hồ sơ xuất viện}: Văn bản lâm sàng mô tả thông tin điều trị bệnh nhân trong một lần nhập viện
	\end{itemize}
	\item \textbf{Bệnh án điện tử:} Bệnh án được số hóa trên máy tính
	\begin{itemize}
		\item Nguồn dữ liệu lớn
		\item Nhiều tri thức tiềm ẩn
	\end{itemize}
	\item Khai thác tri thức trong bệnh án điện tử (BAĐT) đang nhận được nhiều sự quan tâm
	\begin{itemize}
		\item Bước đầu là vấn đề về \emph{rút trích thông tin}
	\end{itemize}		
\end{itemize}
\end{frame}

%\section{Phân giải đồng tham chiếu}
\begin{frame}{Giới thiệu}
\putlogo
\begin{itemize}
	\item Vấn đề cần giải quyết: \textit{Xác định những khái niệm được đề cập trong BAĐT có cùng chỉ tới một thực thể trong thế giới thực hay không}
	\begin{itemize}
		\item[\boldmath$\rightarrow$] Phân giải đồng tham chiếu
	\end{itemize}
	\item Ví dụ:
	\begin{center}
		\tikzstyle{every picture}+=[remember picture]
		\tikzstyle{na} = [shape=rectangle,inner ysep=1mm,inner xsep=0pt]
		``\tikz[baseline=(I.base)]\node[na](I){\color{red}I}; voted for \tikz[baseline=(Nader.base)]\node[na](Nader){\color{blue}Nader}; because \tikz[baseline=(he.base)]\node[na](he){\color{blue}he}; was most\\\addvspace{.5cm}
		aligned with \tikz[baseline=(my.base)]{\node[na](my){\color{red}my};} value'', \tikz[baseline=(she.base)]\node[na](she){\color{red}she}; said
		\begin{tikzpicture}[overlay]
			\path[->,thick] (she) edge[out=150,in=30] (my.north);
			\path[->,thick] (my) edge[out=120,in=300] (I.south);
			\path[->, thick] (he.north west) edge[out=150,in=30] (Nader.north);
		\end{tikzpicture}
	\end{center}
\end{itemize}
\end{frame}

\begin{frame}{Giới thiệu}
\putlogo
\begin{itemize}
	\item \textbf{Bài toán:} Phân giải đồng tham chiếu cho các hồ sơ xuất viện tiếng Anh với các khái niệm đã được trích xuất và gán nhãn
	\item \textbf{Đầu vào:}
	\begin{itemize}
		\item Tập các hồ sơ xuất viện (HSXV)
		\item Tập các khái niệm đã được trích xuất và gán nhãn
%		\begin{itemize}
%			\item Person
%			\item Problem/Test/Treatment
%			\item Pronoun
%		\end{itemize}
	\end{itemize}	
	\item \textbf{Đầu ra:} Danh sách các chuỗi đồng tham chiếu cho mỗi HSXV	
\end{itemize}

\uncover<2>{
\begin{center}
Ý nghĩa các nhãn khái niệm\vspace{0.1cm}
{
\footnotesize
\begin{tabular}{lll}
\toprule
\textbf{Nhãn} & \textbf{Ý nghĩa} & \textbf{Ví dụ}\\
\midrule
\textit{Person} & Các khái niệm chỉ người & the patient, Dr. John\\
\textit{Problem} & Các vấn đề về sức khỏe & \\
\textit{Test} & Các thủ tục y tế & \\
\textit{Treatment} & Các phương pháp điều trị & \\
\textit{Pronoun} & Các đại từ & which, who, that\\
\bottomrule
\end{tabular}
}
\end{center}
}
\end{frame}
