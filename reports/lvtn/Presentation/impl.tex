\section{Hiện thực hệ thống}
%the problem
\subsection{Nội dung bài toán}
\begin{frame}{Nội dung bài toán}
\putlogo
Phân giải đồng tham chiếu cho các hồ sơ xuất viện tiếng Anh với các khái niệm đã được trích xuất và gán nhãn

Đầu vào:
\begin{itemize}
	\item Tập các hồ sơ xuất viện (HSXV)
	\begin{itemize}
		\item Những văn bản lâm sàng mô tả thông tin điều trị bệnh nhân
	\end{itemize}
	\item Tập các khái niệm đã được trích xuất và gán nhãn
	\begin{itemize}
		\item Person
		\item Problem/Test/Treatment
		\item Pronoun
	\end{itemize}
\end{itemize}

Đầu ra: 
\begin{itemize}
\item Danh sách các chuỗi đồng tham chiếu cho mỗi HSXV
\end{itemize}
\end{frame}

%system architecture
\subsection{Kiến trúc hệ thống}
\begin{frame}{Kiến trúc hệ thống}
\putlogo
Mô hình được sử dụng: mô hình \emph{cặp khái niệm}
\begin{itemize}
	\item Xác định tính đồng tham chiếu cho từng cặp hai khái niệm
	\item Gom cụm các cặp đồng tham chiếu lại và xây dựng các chuỗi đồng tham chiếu ứng với mỗi cụm
\end{itemize}
Hai quy trình
\begin{itemize}
	\item Quy trình huấn luyện
	\begin{itemize}
		\item Huấn luyện các mô hình phân loại dựa trên dữ liệu mẫu 
	\end{itemize}
	\item Quy trình phân giải
	\begin{itemize}
		\item Cho các BAĐT mới
		\item Sử dụng các mô hình phân loại đã được huấn luyện
		\item Giải thuật gom cụm
	\end{itemize}
\end{itemize}
\end{frame}

\begin{frame}{Kiến trúc hệ thống}
\putlogo
Trình bày rõ 2 quy trình...
\end{frame}

%pair construction
\subsection{Xây dựng các cặp khái niệm}
\begin{frame}{Xây dựng các cặp khái niệm}
\putlogo
\begin{itemize}
\item Trình bày về sự mất cân bằng mẫu
\item Các cách giải quyết
\end{itemize}
\end{frame}

%person extraction
\subsection{Đặc trưng lớp Person}
\begin{frame}{Đặc trưng lớp Person}
\putlogo
\begin{itemize}
	\item Thường là tên riêng, chức danh và đại từ nhân xưng
	\item Gồm 3 nhóm chính: bệnh nhân, người thân bệnh nhân và nhân sự bệnh viện
	\begin{itemize}
		\item Nhóm bệnh nhân có số lượng khái niệm nhiều nhất
		\item Xác định khái niệm thuộc nhóm nào đóng vai trò quan trọng
	\end{itemize}
	\item Các khái niệm chỉ bệnh nhân được đưa vào cùng một chuỗi đồng tham chiếu
\end{itemize}
\end{frame}

\begin{frame}{Đặc trưng lớp Person}
\begin{itemize}
	\item Có thể được phân loại dựa vào {\color{red} từ khóa}
	\item Các đặc trưng được sử dụng bao gồm:
	\begin{itemize}
		\item Đặc trưng lớp bệnh nhân
		\item Ngữ nghĩa
		\item Ngữ pháp
		\item Khoảng cách
		\item Từ vựng
	\end{itemize}	
\end{itemize}
\end{frame}

%pronoun extraction
\subsection{Đặc trưng lớp Pronoun}
\begin{frame}{Đặc trưng lớp Pronoun}
\putlogo
\begin{itemize}
	\item Là các khái niệm có thể chỉ về bất kì khái niệm thuộc bốn lớp Person, Problem, Treatment và Test
	\begin{itemize}
		\item Ví dụ: which, this, that,...
	\end{itemize}
	\item Cần được {\color{red} xác định lớp khái niệm} mà đại từ ám chỉ
	\item Đại từ đang xét được ghép chung vào chuỗi đồng tham chiếu của khái niệm gần nhất trước đó có cùng lớp khái niệm
	\item Các nhóm đặc trưng được sử dụng bao gồm:
	\begin{itemize}
		\item Quan hệ
		\item Ngữ nghĩa
		\item Ngữ pháp
		\item Khoảng cách
	\end{itemize}
\end{itemize}
\end{frame}

%problem/treatment/test extraction
\subsection{Đặc trưng lớp Problem/Treatment/Test}
\begin{frame}{Đặc trưng lớp Problem/Treatment/Test}
\putlogo
\begin{itemize}
	\item Là nhóm khái niệm chỉ bất thường sức khỏe, phương pháp chữa trị và thủ tục y tế
	\item Chịu ảnh hưởng mạnh bởi {\color{red} ngữ cảnh văn bản} và chứa nhiều {\color{red} từ ngữ chuyên môn}
	\begin{itemize}
		\item ``Pain in \underline{right leg}'' và ``Pain in \underline{left leg}''
		\item ``Head injury'' và ``Head trauma''
	\end{itemize}
	\item Các đặc trưng về tri thức nhân loại:
	\begin{itemize}
		\item Tiêu đề, từ in đậm và liên kết của {\color{red} Wikipedia}
		\item Tên khái niệm và lớp ngữ nghĩa của {\color{red} UMLS}
	\end{itemize}
\end{itemize}
\end{frame}

\begin{frame}{Đặc trưng lớp Problem/Treatment/Test}
\putlogo
\begin{itemize}
	\item Các đặc trưng về {\color{red} ngữ nghĩa} văn bản:
	\begin{itemize}
		\item Cơ quan trên cơ thể
		\item Vị trí
		\item Thuốc y tế
		\item Chỉ định của thủ tục y tế
		\item Thời gian
		\item Phân đoạn bệnh án
		\item Bổ từ
		\item Thiết bị y tế
		\item Phẫu thuật y tế
	\end{itemize}
	\item Ngoài ra các đặc trưng cơ bản cũng được sử dụng
	\begin{itemize}
		\item Ví dụ nhóm đặc trưng khoảng cách, ngữ nghĩa,...
	\end{itemize}
\end{itemize}
\end{frame}

\subsection{Gom cụm}
\begin{frame}{Gom cụm và xây dựng chuỗi đồng tham chiếu}
\putlogo
Giải thuật gom cụm tốt nhất trước (best-first clustering)
\begin{itemize}
	\item Person/Problem/Test/Treatment
	\begin{itemize}
		\item Chọn ra cặp đồng tham chiếu có độ tin cậy cao nhất (được xuất ra bởi SVM)
	\end{itemize} 
	\item Pronoun
	\begin{itemize}
		\item Xác định lớp chính của đại từ và tạo một cặp gồm đại từ đó và khái niệm thuộc lớp chính ở gần nhất trước đó
	\end{itemize}
	\item Gom các cặp mà có chung một khái niệm lại để tạo thành các cụm và xây dựng chuỗi đồng tham chiếu ứng với mỗi cụm
\end{itemize}
\end{frame}
