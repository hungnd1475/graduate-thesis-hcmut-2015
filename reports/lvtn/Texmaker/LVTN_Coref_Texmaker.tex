\documentclass[11pt,a4paper,twoside]{report}
\usepackage[T1]{fontenc}
\usepackage[utf8]{inputenc}
\usepackage[vietnam]{babel}
\usepackage{amsmath}
\usepackage{amsfonts}
\usepackage{amssymb}
\usepackage{graphicx}
\usepackage[tmargin=1in,bmargin=1in,lmargin=1.4in,rmargin=1in]{geometry}
\usepackage{setspace}
\usepackage{array}
\usepackage{booktabs}
\usepackage{tabularx}
%\usepackage[ruled,lined,linesnumbered]{algorithm2e}
\usepackage{titlesec}
\usepackage{caption}
\usepackage{tikz}
\usepackage{fancyhdr}
\usepackage[unicode,hidelinks]{hyperref}
\usepackage{enumitem}
%\usepackage{hanging}
\usepackage{etoolbox}
\usepackage[linewidth=0.8pt,skipbelow=0pt]{mdframed}
%\usepackage{layout}
%\usepackage{showframe}
\usepackage{listings}
\usepackage{color}
\usepackage{subcaption}

\makeatletter

\pagestyle{fancy}
\onehalfspacing
\usetikzlibrary{
	shapes,arrows,shapes.symbols,shapes.geometric,shadows,
	chains,calc,positioning,shapes.misc
}

\renewcommand{\headrulewidth}{0pt}
\renewcommand{\chaptermark}[1]{\markboth{#1}{}}
%\renewcommand{\sectionmark}[1]{\markright{#1}{}}

\fancyhf{}
\fancyhead[RO]{\sffamily Phân giải đồng tham chiếu trong bệnh án điện tử}
\fancyhead[LE]{\sffamily\leftmark}
\fancyfoot[RO,LE]{\sffamily\thepage}
\fancypagestyle{plain}{
   \fancyhf{}
}

\captionsetup{format=hang,labelsep=period,labelfont=bf,font=sf}
\captionsetup[table]{position=top,justification=raggedright}

\setlength\parindent{3em}
\newlength\titleindent
\setlength\titleindent{\parindent}
\titleformat{\section}
  {\normalfont\Large\bfseries}
  {\makebox[\titleindent][l]{\thesection}}
  {0pt}
  {}
  
\tikzset{
	base/.style = { draw, on grid, align=center, on chain, inner sep=2pt },
	doc/.style = { base, shape=tape, fill=white, tape bend top=none, minimum height=2.5em, text width=3em },
	multidoc/.style = { doc, double copy shadow={shadow xshift=.5ex, shadow yshift=.5ex} },
	proc/.style = { base, rectangle, text width=4em, minimum height=2em },
	io/.style = { base, trapezium, trapezium left angle=60, trapezium right angle=120, text width=4em, minimum height=2em },
	altproc/.style = { base, rounded rectangle, text width=3.5em, minimum height=2em },	
	wpoint/.style = { circle, draw, minimum size=0.25cm, inner sep=0cm, outer sep=0cm },
	bpoint/.style = { wpoint, fill=black }, 
	swpoint/.style={wpoint,minimum size=0.15cm},
	sbpoint/.style={bpoint,minimum size=0.15cm},
}

\setlength{\headheight}{16.5pt}
\setlength{\footskip}{35pt}

\newcommand{\colleft}{\raggedright}
\newcommand{\colright}{\raggedleft}

\newcommand{\ra}[1]{\renewcommand{\arraystretch}{#1}}
\newcolumntype{L}{>{\raggedright\arraybackslash}X}
\newcolumntype{P}[2]{>{#1\arraybackslash}p{#2\textwidth}}

\newcommand{\rowgroup}[1]{\hspace{-1em}#1}

%\renewcommand{\thealgocf}{}

%\newenvironment{titlepara}[1]
%{	
%	%\addvspace{0.5\baselineskip}
%	\par\noindent\emph{#1}
%	\begin{hangparas}{3em}{0}
%	\setlength{\parskip}{0.5\baselineskip}
%	\let\svpar\par
%	\edef\svparskip{\the\parskip}
%	\def\revertpar{\svpar\setlength\parskip{\svparskip}\let\par\svpar}
%	\def\noparskip{\leavevmode\setlength\parskip{0pt}%
%  		\def\par{\svpar\let\par\revertpar}}
%	\noparskip\par
%}
%{
%	\end{hangparas}
%	\addvspace{0.5\baselineskip}
%	\par\@afterindentfalse\@afterheading
%}

\newcommand*\NoIndentAfterEnv[1]{%
  \AfterEndEnvironment{#1}{\par\@afterindentfalse\@afterheading}}
  
\NoIndentAfterEnv{itemize}
\NoIndentAfterEnv{enumerate}

\setlist[itemize]{leftmargin=\parindent,parsep=0pt,partopsep=0pt}
\setlist[enumerate]{leftmargin=\parindent,parsep=0pt,partopsep=0pt}

\DeclareMathOperator{\BigO}{O}
\DeclareMathOperator{\Kernel}{\mathbf{K}}
\renewcommand{\vec}[1]{\mathbf{#1}}

\newenvironment{newtable}[2]
{
	\begin{table}[t!]
	\centering\ra{#1}
	\caption{#2}
	\footnotesize\sffamily
}
{
	\end{table}
}

\addto\captionsvietnam{
	\renewcommand{\listfigurename}{Danh sách hình}
	\renewcommand{\listtablename}{Danh sách bảng}
	\renewcommand{\bibname}{Tài liệu tham khảo}
}

\newcolumntype{+}{>{\global\let\currentrowstyle\relax}}
\newcolumntype{^}{>{\currentrowstyle}}
\newcommand{\rowstyle}[1]{\gdef\currentrowstyle{#1}%
#1\ignorespaces
}

\newenvironment{rtable}[2][ht]
{
	\@float{table}[#1]
	\renewcommand{\tabcolsep}{3.75pt}
	\centering\ra{1.2}
	\caption{#2}
	\footnotesize\sffamily
	
	\tabular{@{}llllclllclllclll@{}}
	\toprule
	&\multicolumn{3}{l}{\textbf{MUC}}&\phantom{a}&\multicolumn{3}{l}{\textbf{B-CUBED}}&\phantom{a}&\multicolumn{3}{l}{\textbf{CEAF}}&\phantom{a}&\multicolumn{3}{l}{\textbf{Trung bình}}\\
	\cmidrule{2-4} \cmidrule{6-8} \cmidrule{10-12} \cmidrule{14-16}
	& \textbf{P} & \textbf{R} & \textbf{F} && \textbf{P} & \textbf{R} & \textbf{F} && \textbf{P} & \textbf{R} & \textbf{F} && \textbf{P} & \textbf{R} & \textbf{F}\\
	\midrule
}
{
	\bottomrule
	\endtabular
	\end@float
}

\makeatother

\begin{document}

%\SetKw{Null}{null}
%\SetKw{KwOr}{or}
%\SetKw{KwAnd}{and}
%\SetKw{KwBreak}{break}
%\SetKw{KwIn}{in}
%\SetKw{False}{false}
%\SetKw{True}{true}
%\SetKwInOut{Input}{Đầu vào}
%\SetKwInOut{Output}{Đầu ra}
%\SetKw{KwDownTo}{downto}
%\SetKw{KwContinue}{continue}
%\SetAlgorithmName{Giải thuật}{giải thuật}{Danh sách giải thuật}

\chapter*{Lời cam đoan}
Chúng tôi xin cam đoan rằng, ngoại trừ các kết quả tham khảo từ các công trình khác như đã ghi rõ trong luận văn, các nội dung trình bày trong luận văn này là do chính chúng tôi thực hiện và chưa có phần nội dung nào của luận văn này được nộp để lấy bằng cấp ở một trường khác.

\begin{flushright}
Tp.HCM ngày 17 tháng 12 năm 2015
\end{flushright}

\chapter*{Lời cảm ơn}
Trước hết, chúng tôi xin gửi lời cám ơn chân thành nhất đến GS.TS Cao Hoàng Trụ, giảng viên hướng dẫn luận văn và là người thầy gắn bó với chúng tôi trong nhóm nghiên cứu khoa học hơn một năm vừa qua. Chính nhờ những tri thức thầy truyền đạt cùng với sự hướng dẫn tận tình, những góp ý khoa học của thầy đã giúp chúng tôi hoàn thành tốt nhất luận văn.

Chúng tôi cũng gửi lời cảm ơn đến anh Đào Trọng Điệp, anh Đinh Quang Tuấn, anh Huỳnh Minh Huy, những người đã góp ý cho chúng tôi hoàn thành luận văn tốt nghiệp này. Những tài liệu tham khảo quý báu của các anh là nhân tố không thể thiếu giúp chúng tôi vượt qua khó khăn trong quá trình hiện thực luận văn.

Ngoài ra, chúng tôi cảm ơn tổ chức I2B2 đã cung cấp chúng tôi tập dữ liệu tiếng Anh để thực hiện luận văn. Đồng thời, chúng tôi cảm ơn quý Thầy Cô khoa Khoa học và Kỹ thuật Máy tính trường Đại học Bách Khoa, TP.HCM đã tận tình dạy dỗ chúng tôi suốt hơn bốn năm qua.

Cuối cùng, chúng tôi xin cảm ơn gia đình, bạn bè, những người đã hỗ trợ chúng tôi về mặt tinh thần và giúp đỡ chúng tôi trong suốt quá trình thực hiện đề tài này.

\chapter*{Tóm tắt}
Bệnh án điện tử là vấn đề đang nhận được nhiều sự quan tâm nghiên cứu trên thế giới. Các nước phát triển đã đạt nhiều thành tựu trong việc xây dựng và sử dụng bệnh án điện tử vào khám chữa bệnh. Tại Việt Nam, các bệnh viện lớn như Chợ Rẫy, Hùng Vương,... đã bắt đầu sử dụng các hệ thống bệnh án điện tử, tuy nhiên các hệ thống này vẫn còn đơn giản và chưa khai thác hoàn toàn thông tin trong bệnh án. Bài toán phân giải đồng tham chiếu hiện đang được nhiều tổ chức nghiên cứu ứng dụng vào bệnh án điện tử, trong đó có tổ chức I2B2. Vì vậy, chúng tôi thực hiện đề tài \textit{Phân giải đồng tham chiếu trong hồ sơ xuất viện tiếng Anh} nhằm tạo tiền đề cho các nghiên cứu khác trong bệnh án điện tử tại Việt Nam.

Về phương pháp thực hiện, chúng tôi xây dựng hệ thống dựa trên bài báo \textit{A classification approach to coreference in discharge summaries: 2011 I2B2 challenge}. Hệ thống được hiện thực sử dụng mô hình cặp khái niệm trong phân giải đồng tham chiếu, kết hợp với phương pháp học máy sử dụng mô hình Support Vector Machine. Kết quả được thử nghiệm trên tập dữ liệu của tổ chức I2B2 bao gồm 251 hồ sơ xuất viện cho tập huấn luyện và 173 hồ sơ xuất viện cho tập kiểm tra. Mô hình của chúng tôi đạt F-measure trung bình là 81.5\% khi tính trên ba độ đo MUC, B-CUBED và CEAF. Kết quả nghiên cứu hứa hẹn việc áp dụng hệ thống vào sử dụng thực tiễn.



\tableofcontents{}

\clearpage
\phantomsection
\addcontentsline{toc}{chapter}{\listfigurename}
\listoffigures

\clearpage
\phantomsection
\addcontentsline{toc}{chapter}{\listtablename}
\listoftables

\chapter{Tổng quan}
\section{Giới thiệu đề tài}\label{gioithieudetai}
Trong hơn mười năm trở lại đây, với sự bùng nổ của kỉ nguyên công nghệ thông tin, việc số hóa dữ liệu trở nên phổ biến hơn bao giờ hết, và bệnh án cũng không phải là ngoại lệ. Bệnh án điện tử (Electronic Medical Record) đã và đang dần thay thế cho phương pháp ghi chép và lưu trữ truyền thống thông tin của bệnh nhân trong quá trình khám và chữa bệnh. Hầu hết bệnh viện ở những nước phát triển đã triển khai các hệ thông tin bệnh viện (HTTBV) để phục vụ cho việc số hóa loại tài liệu này.

Bên cạnh việc xây dựng bệnh án điện tử (BAĐT) thì việc khai thác nguồn dữ liệu lớn này cũng là một lĩnh vực đang rất được quan tâm trong những năm gần đây. Năm 2004, Viện y tế Quốc gia Hoa Kì (NIH: National Institute of Health) đã kêu gọi thành lập mạng lưới nghiên cứu cấp quốc gia về y sinh. Để đáp lại lời kêu gọi đó, bảy Trung tâm nghiên cứu công nghệ tính toán y sinh (NBCB: National Center for Biomedical Computing) đã được thành lập dưới sự tài trợ của NIH với nhiệm vụ xây dựng cơ sở hạ tầng phục vụ cho việc áp dụng khoa học máy tính vào lĩnh vực y sinh, hỗ trợ cho công việc nghiên cứu. Trong đó, i2b2 (Informatics for Integrating Biology and the Bedside), một NBCB được thành lập bởi sự hợp tác giữa hai trường đại học nổi tiếng Havard và MIT, bắt đầu từ năm 2006 đã tổ chức các cuộc thi hàng năm nhằm tìm kiếm các phương pháp phân tích và rút trích kiến thức trên dữ liệu BAĐT, gọi tắt là các Thách thức (Challenges). Mỗi Thách thức đưa ra một vấn đề phân tích và một tập dữ liệu BAĐT được cung cấp bởi các bệnh viện trong và ngoài nước Mỹ. Hàng năm có trên dưới 100 nhóm nghiên cứu tham gia đề xuất giải pháp và gửi kết quả phân tích, những giải pháp tốt được chọn lọc để công bố ở một hội thảo quốc tế và được áp dụng rộng rãi vào các dịch vụ chăm sóc sức khỏe.

Tại Việt Nam, các HTTBV cũng đang dần được triển khai, tiêu biểu là Bệnh viện đa khoa Vân Đồn tỉnh Quảng Ninh – cơ quan y tế đầu tiên có trang bị hệ thống bệnh án điện tử hiện đại và hoàn chỉnh với giải pháp MEDI SOLUTIONS của công ty phần mềm Hoa Sen. Cùng với việc xây dựng, tập thể nghiên cứu ``Học máy và ứng dụng'' của viện John von Neumann thuộc đại học Quốc Gia TP Hồ Chí Minh đã tiến hành phát triển các phương pháp và phần mềm phục vụ cho khai thác bệnh án điện tử tiếng Việt. Một trong những vấn đề của việc khai thác dữ liệu BAĐT đó là phân giải đồng tham chiếu. Thách thức lần thứ 5 (năm 2011) của i2b2 đã đưa ra một cái nhìn có hệ thống về vấn đề này. Một cách tổng quát, việc phân giải đồng tham chiếu các khái niệm trong văn bản là xác định liệu hai sự đề cập trong cùng văn bản có ám chỉ tới cùng một sự vật hoặc hiện tượng hay không, từ đó xây dựng các chuỗi đồng tham chiếu. Khi mà đa phần các văn bản được viết tay bằng ngôn ngữ tự nhiên, chứa đựng rất nhiều các khái niệm phụ thuộc vào ngữ cảnh thì việc phân giải đồng tham chiếu này giúp cho máy tính có một cái nhìn mang tính cấu trúc hơn về văn bản, từ đó làm nền tảng cho việc rút trích các kiến thức sâu từ những hiểu biết này.

Tuy vấn đề về phân giải đồng tham chiếu trong những năm gần đây đã được quan tâm nghiên cứu rất nhiều cho các loại văn bản khác (ví dụ các bài báo) thì ở phạm vi BAĐT vấn đề này vẫn còn ít được sự quan tâm. Đứng trước nhu cầu đó, nhóm quyết định bắt tay vào phát triển một hệ thống phân giải đồng tham chiếu cho dữ liệu BAĐT.

\section{Mục tiêu và phạm vi đề tài}
Những ích lợi BADT mang lại như đề cập trong Phần \ref{gioithieudetai} đã tạo động lực cho chúng tôi tiến hành huấn luyện hệ thống phân giải đồng tham chiếu trong BAĐT. Việc xây dựng thành công hệ thống phân giải đồng tham chiếu góp phần hỗ trợ cho các công trình nghiên cứu sâu hơn về BAĐT sau này cũng như được dùng để xây dựng các công cụ thống kê hoặc các hệ thống truy xuất thông tin trong y tế.

Vì giới hạn thời gian, chúng tôi quyết định chỉ huấn luyện hệ thống phân giải đồng tham chiếu cho các hồ sơ xuất viện được viết bằng tiếng Anh với danh sách các thực thể đã được xác định trước. Thách thức I2B2 2011 cung cấp sẵn tập dữ liệu là các báo cáo xuất viện được phân giải đồng tham chiếu sẵn bởi các chuyên gia y tế phù hợp cho quá trình huấn luyện hệ thống. Tập dữ liệu này được tổ chức I2B2 cung cấp miễn phí kèm theo một số cam kết sử dụng dữ liệu (Data agreement), đây là tập dữ liệu vô cùng giá trị cho viết huấn luyện hệ thống. Dựa trên kết quả Thách thức 2011, vấn đề phân giải đồng tham chiếu trong hồ sơ xuất viện có hai hướng giải quyết đạt kết quả tốt là: hướng tiếp cận dựa trên các luật (Rule-based hoặc còn gọi là hướng tiếp cận về ngôn ngữ học) và hướng tiếp cận dựa trên Học máy. Chúng tôi quyết định hiện thực hệ thống dựa trên bài báo có kết quả tốt nhất trong Thách thức 2011 của tác giả Yan Xu \cite{YanXu2012}. Trong đó, giải thuật học máy được áp dụng là Support Vector Machine và mô hình phân giải đồng tham chiếu được áp dụng là mô hình cặp khái niệm (Mention-pair)

\section{Cấu trúc luận văn}
Nội dung bài viết được chúng tôi chia thành bảy chương. Từng chương trình bày một quá trình để xây dựng hoàn thiện hệ thống phân giải đồng tham chiếu trong HSXV cũng như kỹ thuật và kết quả các quá trình hiện thực hệ thống. Chương cuối cùng dùng để tóm tắt, nêu các đánh giá để tìm ra hạn chế và các hướng mở rộng, cải tiến trong tương lai. Nội dung sơ lược của từng chương được tóm tắt như sau.

Chương một nêu lên mục tiêu, động cơ và phạm vi của Luận văn. Toàn bộ chương một giúp người đọc có được cái nhìn toàn cảnh về lí do chúng tôi tiến hành thực hiện đề tài, vai trò và vị trí của đề tài trong việc phát triển BAĐT ở Việt Nam cũng như phạm vi hiện thực của đề tài.

Chương hai giới thiệu các công trình liên quan đến hệ thống. Toàn bộ chương hai giúp người đọc có được kiến thức khái quát về các lĩnh vực nền tảng cho việc phân giải đồng tham chiếu trong BAĐT, đây là các kiến thức nền giúp người đọc dễ dàng nắm bắt quá trình hiện thực hệ thống. Các lĩnh vực này bao gồm khái niệm BAĐT là gì và đầu vào cần thiết trước khi tiến hành phân giải đồng tham chiếu.

Chương ba giới thiệu các kiến thức nền tảng được sử dụng trực tiếp trong việc xây dựng hệ thống. Toàn bộ chương ba giúp người đọc hiểu được các thuật ngữ, kĩ thuật và công nghệ được áp dụng khi tiến hành hiện thực hệ thống. Các kiến thức được nêu chỉ mang tính giới thiệu, nội dung sâu hơn của các kĩ thuật có thể được tìm thấy trong nguồn tham khảo.

Chương bốn giới thiệu chi tiết hệ thống chúng tôi đã xây dựng. Qua chương bốn, người đọc có thể hiểu được kiến trúc tổng quát toàn bộ hệ thống cũng như chi tiết hiện thực từng bước như tiền xử lý, rút trích đặc trưng, huấn luyện mô hình SVM và gom cụm xây dựng chuỗi đồng tham chiếu. Chương bốn giúp người đọc có thể tự hiện thực lại toàn bộ hệ thống với mục đích nghiên cứu, cải tiến hoặc sử dụng.

Chương năm giới thiệu về tập dữ liệu được sử dụng trong quá trình huấn luyện cũng như các thí nghiệm đánh giá hệ thống. Trong chương năm, chúng tôi đề cập chi tiết và các văn bản được cung cấp trong tập dữ liệu I2B2 2011 và các bước để có được tập dữ liệu này. Ngoài ra, các độ đo được sử dụng và kết quả thí nghiệm để đánh giá hiệu năng hệ thống cũng được nêu chi tiết.

Nội dung tổng kết được chúng tôi đề cập trong chương bảy, bao gồm tóm tắt nội dung luận văn, các nhận xét và hướng phát triển trong tương lai.
\chapter{Các công trình liên quan}
\section{Bệnh án điện tử}
Bệnh án là văn bản ghi chép các thông tin sức khỏe của một cá nhân trong quá trình khám và chữa bệnh. Bệnh án điện tử chính là bệnh án được số hóa bằng HTTBV. BAĐT thông thường chứa những dữ liệu cơ bản cho quản lý, các dữ liệu cận lâm sàng và lâm sàng của người bệnh trong một lần nằm viện \cite{HoTuBao2015}. Dữ liệu lâm sàng là những \textit{văn bản lâm sàng} (clinical text) do bác sĩ và y tá ghi chép hàng ngày về thông tin khám và chữa bệnh của người bệnh. Các văn bản lâm sàng trong bệnh án điện tử chủ yếu gồm ba loại \cite{HoTuBao2015}:

\begin{enumerate}[leftmargin=\the\parindent]
\item \emph{Phiếu điều trị} (doctor daily notes): ghi chép các chuẩn đoán, nhận định và y lệnh hàng ngày của bác sĩ về bệnh nhân.
\item \emph{Phiếu chăm sóc} (nurse narratives): là những ghi chép trong ngày của y tá trong quá trình chăm sóc và thực hiện y lệnh của bác sĩ.
\item \emph{Hồ sơ xuất viện} (discharge summary): toàn bộ dữ liệu và thông tin cơ bản của bệnh nhân trong một lần điều trị.
\end{enumerate}

So với bệnh án được lưu trữ bằng giấy, bệnh án điện tử có nhiều ưu điểm như:

\begin{itemize}
\item Lưu trữ chính xác và đầy đủ thông tin bệnh nhân, tránh trùng lặp dữ liệu.
\item Hỗ trợ quá trình tìm kiếm và truy xuất thông tin nhanh chóng.
\item Dữ liệu có thể được chia sẻ hoặc tích hợp.
\end{itemize}

Dữ liệu trong BAĐT thường tồn tại dưới dạng tường thuật, ghi chép của bác sĩ hoặc y tá, đây là dạng dữ liệu không có cấu trúc. Một số thông tin hữu ích có trong dữ liệu không cấu trúc của BAĐT là:

\begin{itemize}
\item Lý do nhập viện, lịch sử điều trị, tiền sử thuốc sử dụng.
\item Ghi chép của bác sĩ hoặc y tá trong quá trình điều trị hàng ngày.
\item Các kết quả xét nghiệm.
\item Các ghi chú về quá trình phẫu thuật.
\end{itemize}

Ngoài các văn bản lâm sàng còn được lưu trữ dưới dạng phi cấu trúc, một số tiêu chuẩn được đưa ra để lưu trữ một cách có cấu trúc các BAĐT:

\begin{itemize}[noitemsep]
\item \emph{IDC} (International Classification of Diseases): bao gồm các loại mã cũng như thông tin về bệnh như tên bệnh, mô tả, triệu chứng, dấu hiệu, mức độ, ...
\item \textit{CPT} (Current Procedural Terminology): bao gồm các mã mang tính thủ tục trong bệnh viện như mã xét nghiệm, gây tê, phẫu thuật, X quang, thuốc, cấp cứu, ...
\end{itemize}

BAĐT hiện đang là vấn đề được quan tâm và phát triển tại nhiều nơi trên thế giới. Vào năm 2009, ngay sau khi trở thành tổng thống Hoa Kỳ, Barack Obama đã yêu cầu chuẩn hóa và số hóa mọi bệnh án của các bệnh viện trong vòng 5 năm. Ở Nhật Bản, các bệnh viện lớn và vừa cũng được chính phủ tạo điều kiện để xây dựng BAĐT. Tính đến năm 2011, khoảng 34.7\% bệnh viện lớn và vừa tại Nhật đã có hệ thống BAĐT sử dụng được \cite{HoTuBao2015}. Một số ví dụ thực tiễn trong việc ứng dụng BAĐT vào dự đoán, điều trị bệnh là hệ thống dự đoán nguy cơ mắc bệnh đái tháo đường loại 2 từ cấu trúc gen \cite{AbelKho2012} hoặc hệ thống cho phép nghiên cứu diện rộng bệnh tâm thần và cách điều trị chứng phiền muộn \cite{Perlis2012}. Lí do vì việc sử dụng BAĐT không những thuận tiện hơn bệnh án giấy trong việc lưu giữ các thông tin và tri thức thu thập được trong quá trình khám chữa bệnh mà còn cho phép chia sẻ nguồn thông tin đó giữa các bệnh viện, các thành phố hoặc giữa các quốc gia với nhau. Thông qua việc chia sẻ, rất nhiều BAĐT sẽ được đối chiếu và phân tích để phát hiện những tri thức y học mới hoặc kiểm chứng những kiến thức đã có. Vì vậy BAĐT đóng vai trò quan trọng trong sự phát triển của việc khám chữa bệnh cũng như nghiên cứu trong y học.

\section{Nhận dạng thực thể có tên}
Rút trích thông tin (information extraction) là công việc tự động rút trích thông tin từ những dữ liệu không có cấu trúc hoặc dữ liệu bán cấu trúc. Dữ liệu có cấu trúc là dữ liệu máy tính hiểu hoàn toàn. dữ liệu có cấu trúc thông thường nằm ở dạng bảng, hoặc trong các hệ quản trị dữ liệu quan hệ. Dữ liệu không có cấu trúc là dữ liệu máy hoàn toàn không hiểu, như ngôn ngữ tự nhiên. Dữ liệu bán cấu trúc là dữ liệu chứa các thẻ hoặc các hình thức đánh dấu khác giúp phân tách bộ phận ngữ cảnh nền ra khỏi dữ liệu, điển hình là các ngôn ngữ đánh dấu như XML, JSON, HTML. Rúc trích thông tin là một trong những vấn đề của xử lý ngôn ngữ tự nhiên.

Tác vụ rút trích thông tin gồm hai bước con là nhận dạng thực thể và rút trích quan hệ. Trong đó, nhận dạng thực thể là bước đầu tiên của bài toán rút trích thông tin. Nhận dạng thực thể có tên là xác định các thực thể có trong văn bản và phân loại chúng vào các lớp khái niệm được định nghĩa sẵn, trong đó khái niệm có tên là các cụm từ có chứa tên của người, tổ chức hoặc địa điểm \cite{KimSang2003}. Ví dụ đầu ra của bước nhận dạng thực thể từ câu văn ``Duy Hưng là sinh viên đại học Bách Khoa của thành phố Hồ Chí Minh'' là:

\begin{itemize}
\item ``Duy Hưng'' - Con người
\item ``đại học Bách Khoa'' - Tổ chức
\item ``thành phố Hồ Chí Minh'' - Nơi chốn
\end{itemize}

Bài toán nhận dạng thực thể có tên thường bao gồm 2 bước: xác định thực thể và phân loại thực thể vào các nhóm ngữ nghĩa (như con người, nơi chốn, tổ chức, ...) \cite{KimSang2003}. Trong đó, bước đầu tiên của bài toán thường được xem đơn giản như là một bài toán phân mảnh các từ trong câu thành các ``tên'', với ``tên'' là một chuỗi các từ liên tục có ý nghĩa và chỉ tới một thực thể có thật, không lồng ghép trong ``tên'' khác. Vì vậy từ ``Ngân hàng Việt Nam'' chỉ được xem như là một tên duy nhất, mặc dù từ ``ngân hàng'' và ``Việt Nam'' bản thân cũng mang ý nghĩa.

Các hệ thống nhận diện thực thể có tên nếu hoạt động tốt trong một lĩnh vực cụ thể (như y tế, địa chất, ký sự, ...) thì sẽ cho kết quả không tốt nếu đem ứng dụng vào lĩnh vực khác. Việc chỉnh sửa cho một hệ thống có sẵn hoạt động tốt trong một lĩnh vực mới thường tiêu tốn rất nhiều công sức.

Tùy theo mỗi lĩnh vực quan tâm cụ thể, các loại thực thể sẽ được định nghĩa khác nhau. Với những vấn đề không đặc thù, những nhóm thực thể thường được nhắc đến là: động vật, người, tổ chức, vật, ... Khi nghiên cứu về nhận dạng thực thể trong bệnh án điện tử, năm loại thực thể cần được quan tâm là: vấn đề (Problem), phương pháp điều trị (Treatment), các xét nghiệm (Test), con người (Person) và đại từ (Pronoun).

Năm 2010, trung tâm i2b2 đưa ra Thách thức về vấn đề xử lý ngôn ngữ tự nhiên cho các văn bản y tế lâm sàng bao gồm ba tác vụ:

\begin{enumerate}
\item Trích xuất và nhận dạng các thực thể có tên trong y học.
\item Phân loại bệnh vào một trong các dạng: đang xảy ra ở hiện tại, không xảy ra ở hiện tại, có thể xảy ra trong tương lai, ...
\item Rút trích các quan hệ giữa các bệnh, phương pháp điều trị và thủ tục y tế.
\end{enumerate}

Đối với Thách thức này, i2b2 tập trung vào giải quyết nhóm bài toán rút trích thông tin vì đây là nhóm bài toán nền tảng, tạo tiền đề để nghiên cứu cho các hướng đi khác. Tuy Thách thức i2b2 năm 2010 có đề cập đến việc rút trích các quan hệ giữa các thực thể trong bệnh án (tác vụ thứ 3), nhưng mối quan hệ đồng tham chiếu lại không được bao gồm trong số đó. Chính vì thế, năm 2011 i2b2 tổ chức Thách thức lần thứ 5 dành riêng cho việc giải quyết vấn đề phân giải đồng tham chiếu trên dữ liệu BAĐT với đầu vào là kết quả nhận diện thực thể từ Thách thức năm 2010. Vấn đề được nêu trong Thách thức i2b2 2011 cũng chính là vấn đề được chúng tôi giải quyết trong nội dung luận án mà chúng tôi trình bày chi tiết trong các phần sau.
\chapter{Kiến thức nền tảng}
\section{Các định nghĩa và thuật ngữ}
\section{Support Vector Machine}
\section{Các mô hình học máy phân giải đồng tham chiếu}
\section{Các công cụ hỗ trợ rút trích đặc trưng}

\chapter{Chi tiết hệ thống}
\section{Tiền xử lý}
Trong quá trình rút trích đặc trưng, một số khái niệm được miêu tả cụ thể làm cho việc so trùng chuỗi hoặc tìm kiếm từ các nguồn tri thức nhân loại thiếu chính xác \cite{YanXu2012}. Ví dụ như khái niệm "her CT scan" và khái niệm "a CT scan". Mặc dù hai khái niệm này cùng chỉ một thủ tục y tế nhưng không trùng chuỗi. Ngoài ra các mạo từ "her", "a" làm việc tìm kiếm tri thức nhân loại từ các nguồn tri thức như Wikipedia, WordNet không được chính xác hoặc không thể tìm được kết quả. Vì vậy trước khi rút trích đặc trưng, các khái niệm được tiền xử lý để loại bỏ mạo từ và các thông tin ngữ cảnh. Tuy nhiên, quá trình tiền xử lý chỉ được áp dụng cho các đặc trưng liên quan so trùng chuỗi và tìm kiếm tri thức nhân loại, các đặc trưng khác không cần qua quá trình tiền xử lý mà nhận vào nguyên gốc khái niệm được xác định.

Quá trình tiền xử lý gồm hai bước. Đầu tiên khái niệm sẽ được loại bỏ tất cả mạo từ. Sau đó, nếu khái niệm có bao gồm giới từ thì giới từ đó và toàn bộ nội dung theo sau sẽ được lược bỏ. Ví dụ như khái niệm “an MRI of the knee” sau quá trình tiền xử lý sẽ trở thành “MRI”. Danh sách mạo từ được xây dựng từ tập dữ liệu và các mạo từ thông dụng của tiếng Anh.

Đặc biệt các khái niệm thuộc lớp Problem/Treatment/Test thường được kèm thêm thông tin về định lượng như 10mg, 5 lit và các thông tin về vị trí giải phẫu học như "upper", "left", "right". Để tăng khả năng tìm kiếm tri thức nhân loại, chúng tôi đề xuất loại bỏ các thông tin ngữ cảnh về số, định lượng và vị trí giải phẫu khỏi khái niệm. Các thông tin ngữ cảnh được loại bỏ bằng cách sử dụng biểu thức chính quy và các từ vựng được xây dựng từ tập dữ liệu. Các đặc trưng liên quan so trùng chuỗi không áp dụng bước tiền xử lý loại bỏ thông tin ngữ cảnh này.
\section{Xây dựng các cặp khái niệm}
\section{Rút trích đặc trưng}
Từ các phân tích được đề cập ở Phần 3, ngoài các thuộc tính chung về mặt ngôn ngữ (như ngữ pháp hay từ vựng), từng lớp khái niệm ở BAĐT còn mang những đặc tính khác nhau. Việc này đòi hỏi chúng tôi phải thiết kế ba hệ thống rút trích đặc trưng và phân loại tương ứng khác nhau cho lớp Person, lớp Problem/Treatment/Test và lớp Pronoun. Hình \ref{fig:TongquanPhangiai} mô tả tổng quan ba hệ thống này, trong đó các khối "Đồng tham chiếu lớp X" bao hàm cả Hệ thống rút trích đặc trưng và Hệ thống phân loại cho lớp tương ứng.

\begin{figure}[ht]
\centering
\begin{tikzpicture}[%
	>=angle 60,
	start chain=going below,
	node distance=1.5cm and 4cm,
	every join/.style={->, draw},
	font=\tiny\sffamily]
	\tikzset{
		wproc/.style = {proc, text width=4.5em},
		wdoc/.style = {doc, text width=3.5em}
	};
	
	\node[io](pati){Các khái niệm Person};
	\node[io](perp){Các cặp khái niệm Person};
	\node[io](prop){Các cặp khái niệm Problem};
	\node[io](tstp){Các cặp khái niệm Test};
	\node[io](trep){Các cặp khái niệm Treatment};
	\node[io](pron){Các khái niệm Pronoun};
	
	\node[wdoc,left=2cm of $(prop)!0.5!(tstp)$](emr){EMR + Concepts};
	\node[altproc,right=2.5cm of pati](patic){Phân loại bệnh nhân};
	\node[wproc,right=4cm of perp](perr){Phân giải đồng tham chiếu lớp Person};
	\node[wproc](pror){Phân giải đồng tham chiếu lớp Problem};
	\node[wproc](tstr){Phân giải đồng tham chiếu lớp Test};
	\node[wproc](trer){Phân giải đồng tham chiếu lớp Treatment};
	\node[wproc](pronr){Phân giải đồng tham chiếu lớp Pronoun};

	\node[io,right=6cm of perr](perch){Các chuỗi Person};
	\node[io](proch){Các chuỗi Problem};
	\node[io](tstch){Các chuỗi Test};
	\node[io](trech){Các chuỗi Treatment};
	
	\coordinate[left=0.5cm of perch.west](perch-left);
	\coordinate[left=1cm of proch.west](proch-left);
	\coordinate[left=1.5cm of tstch.west](tstch-left);
	\coordinate[left=2cm of trech.west](trech-left);
	
	\draw[->] (emr) -- ++(right:1.3cm) |- (pati);
	\draw[->] (emr) -- ++(right:1.3cm) |- (perp);
	\draw[->] (emr) -- ++(right:1.3cm) |- (prop);
	\draw[->] (emr) -- ++(right:1.3cm) |- (tstp);
	\draw[->] (emr) -- ++(right:1.3cm) |- (trep);
	\draw[->] (emr) -- ++(right:1.3cm) |- (pron);
	\draw[->] (pati) -> (patic);
	\draw[->] (patic) -| (perr);
	\draw[->] (perp) -> (perr);
	\draw[->] (prop) -> (pror);
	\draw[->] (tstp) -> (tstr);
	\draw[->] (trep) -> (trer);
	\draw[->] (pron) -> (pronr);
	\draw[->] (perr) -> (perch);
	\draw[->] (pror) -> (proch);
	\draw[->] (tstr) -> (tstch);
	\draw[->] (trer) -> (trech);
	\draw[->] (pronr) -| (perch-left);
	\draw[->] (pronr) -| (proch-left);
	\draw[->] (pronr) -| (tstch-left);
	\draw[->] (pronr) -| (trech-left);
\end{tikzpicture}
\caption{Tổng quan hệ thống phân giải đồng tham chiếu \label{fig:TongquanPhangiai}}
\end{figure}

\subsection*{Nhóm Person}
Tổng quát, các khái niệm thuộc lớp Person có thể là các đại từ nhân xưng (he, she, it, they, ...), đại từ sở hữu (his, her, its, their, ...), đại từ phản thân (himself, herself, itself, themselves, ...) hoặc tên người (Stephanie I Sept, Mr. Anders, Heidi Laura Md, ...). Việc phân giải đồng tham chiếu cho tên người và đại từ là công việc khó, vì thông tin có được từ các đại từ và tên người là rất ít. Ngoài ra trong một văn bản thường đề cập đến nhiều hơn một người, khiến cho việc phát hiện chính xác chuỗi đồng tham chiếu cho các khái niệm này là một thách thức lớn.

Dựa vào hệ thống I, việc giới hạn vấn đề lại trong phạm vi BAĐT giúp công việc này trở nên đơn giản hơn. Trong BAĐT, các khái niệm thuộc lớp Person thường được chia vào ba nhóm chính: bệnh nhân, người thân của bệnh nhân hoặc nhân sự của bệnh viện. Trong đó bệnh nhân là nhóm có số lượng khái niệm được đề cập nhiều nhất và chiếm phần lớn tổng số khái niệm lớp Person. Do vậy việc xác định một khái niệm thuộc vào nhóm nào đóng vai trò quan trọng trong việc phân giải chính xác chuỗi đồng tham chiếu cho khái niệm đó \cite{YanXu2012}. Từ lí do trên, đặc trưng có phải là bệnh nhân hay không được thêm vào hệ thống. Đặc trưng lớp Patient được xác định bằng phương pháp phân loại nhị phân SVM. Hai nhóm người thân của bệnh nhân và nhân sự của bệnh viện được xác định bằng các đặc trưng từ vựng. Bảng \ref{tab:PersonFeatures} trình bày đầy đủ các đặc trưng dùng cho lớp Person.

\begin{table}[th]
\centering\ra{1.2}
\caption{Tập đặc trưng cho lớp Person \label{tab:PersonFeatures}}
\footnotesize\sffamily

\begin{tabularx}{\textwidth}{@{}P{\raggedright}{0.3}lL@{}}
\toprule 
\textbf{Đặc Trưng} & \textbf{Giá trị} & \textbf{Giải thích}\\
\midrule
Patient-class & 0, 1, 2 & Không có khái niệm nào là bệnh nhân (0), cả hai khái niệm đều là bệnh nhân (1), trường hợp khác (2)\\
Distance between sentences & 0, 1, 2, 3, ... & Số câu xuất hiện giữa hai khái niệm được xét\\
Distance between mentions & 0, 1, 2, 3, ... & Số khái niệm xuất hiện giữa hai khái niệm được xét\\
String match & 0, 1 & Trùng chuỗi hoàn toàn (1), ngược lại (0)\\
Levenshtein distance between two mentions & 0, 1, 2, 3, ... & Khoảng cách Levenshtein giữa hai khái niệm\\
Number & 0, 1, 2 & Cả hai đều là số ít hoặc nhiều (1), ngược lại (0), không xác định (2)\\
Gender & 0, 1, 2 & Cùng giới tính (1), khác giới tính (0), không xác định (2)\\
Apposition & 0, 1 & Là đồng vị ngữ (1), ngược lại (0)\\
Alias & 0, 1 & Là từ viết tắt hoặc cùng nghĩa (1), ngược lại (0)\\
Who & 0, 1 & Nếu hai khái niệm liền kề nhau và được phân cách bởi dấu ``:''\\
Name match & 0, 1 & Loại bỏ các	``stop word'' (dr, dr., mr, ...), so trùng chuỗi con, trùng (1), không trùng (0)\\
Relative match & 0, 1 & Cả hai đều cùng chỉ đến một thân nhân (1), ngược lại (0)\\
Department match & 0, 1 & Cả hai cùng chỉ đến một lĩnh vực y học (1), ngược lại (0)\\
Doctor title match & 0, 1 & Cả hai có cùng một chức vụ bác sĩ (1), ngược lại (0)\\
Doctor general match & 0, 1 & Cả hai cùng đề cập đến bác sĩ nói chung (1), ngược lại (0)\\
Twin/triplet & 0, 1 & Cả hai đều chỉ về cùng cặp sinh đôi/sinh ba (1), ngược lại (0)\\
We & 0, 1 & Cả hai đều chứa thông tin về ``chúng tôi'' (1), ngược lại (0)\\
You & 0, 1 & Cả hai đều chứa thông tin về ``tôi'' (1), ngược lại (0)\\
I & 0, 1 & Cả hai đều chứa thông tin về ``bạn'' (1), ngược lại (0)\\
Pronoun match & 0, 1 & Cả hai đều là đại từ chỉ người (1), ngược lại (0)\\
\bottomrule
\end{tabularx}
\end{table}

Với các đặc trưng ``Name match'', ``Relative match'', ``Department match'', ``Doctor title match'', ``Doctor general match'', ``Twin/Triplet'', ``We'', ``You'', ``I'', ``Pronoun match'', chúng tôi hiện thực bằng cách xây dựng tập từ điển tương ứng với từng đặc trưng dựa trên việc khảo sát tập dữ liệu và sử dụng các biểu thức chính quy.

Đặc trưng về Giới tính được chúng tôi xác định dựa trên ba bước phân loại \cite{WeeSoon2001}. Bước thứ nhất: kiểm tra khái niệm có chứa các đại từ xác định giới tính như ``Mr'', ``Ms'', ``she'', ``he'', ... hay không. Nếu có, xác định giới tính dựa trên đại từ xuất hiện. Nếu không thực hiện bước thứ hai: kiểm tra khái niệm có xuất hiện nhiều hơn một lần hay không. Nếu xuất hiện nhiều hơn một lần thì các lần xuất hiện có chứa đại từ xác định giới tính hay không. Ví dụ khái niệm ``Peter H. Diller'' có thể xuất hiện nhiều lần, trong đó có xuất hiện dưới hình thức ``Mr. Diller''. Nếu không thể xác định giới tính qua hai bước kiểm tra, khái niệm sẽ được phân loại bằng cách sử dụng cơ sở dữ liệu về tên tiếng Anh của hệ thống Apache OpenNLP.

\subsection*{Nhóm Patient-class}
Từ nhận định trong việc rút trích đặc trưng của lớp Person, chúng tôi xây dựng một hệ thống SVM nhị phân để phân loại khái niệm thuộc lớp Person có phải là bệnh nhân hay không. Trong BAĐT thường chỉ có một bệnh nhân đóng vai trò là chủ thể của bệnh án.Vì vậy, các khái niệm nếu được xác định là bệnh nhân, thì sẽ được đưa vào một chuỗi đồng tham chiếu duy nhất về bệnh nhân đó. Thông qua phân tích tập dữ liệu, chúng tôi nhận thấy việc xác định một khái niệm thuộc lớp Person hay không có thể đạt được bằng cách xác định tập từ khóa chỉ về bệnh nhân như ``patient'', ``pt'', ... và tập từ khóa chỉ về nhóm người không phải bệnh nhân như ``doctor'', ``dr'', ``wife'', ...

Vì tập dữ liệu không có thông tin xác định một khái niệm thuộc lớp Person có phải là bệnh nhân hay không, dựa theo hệ thống I chúng tôi xác định bằng cách chọn chuỗi đồng tham chiếu có nhiều khái niệm nhất trong tập kết quả làm chuỗi đồng tham chiếu chỉ bệnh nhân. Các khái niệm thuộc chuỗi đồng tham chiếu này sẽ được xem là khái niệm chỉ bệnh nhân và được chọn làm mẫu dương trong quá trình huấn luyện. Các khái niệm thuộc lớp Person còn lại không thuộc vào chuỗi đồng tham chiếu này sẽ được chọn làm mẫu âm trong quá trình huấn luyện. Tuy nhiên, chúng tôi nhận thấy phương pháp xác định bệnh nhân này có một nhược điểm là các BAĐT nhỏ, có nội dung ngắn sẽ tồn tại nhiều chuỗi đồng tham chiếu lớp Person có kích thước tương tự nhau. Trong đó chuỗi đồng tham chiếu chỉ bệnh nhân không chắc chắn là chuỗi đồng tham chiếu có kích thước lớn nhất.

Bảng \ref{tab:PatientFeatures} trình bày đầy đủ các đặc trưng được sử dụng cho việc xác định khái niệm có phải là bệnh nhân hay không. Kết quả của việc phân loại này sẽ được sử dụng làm giá trị cho đặc trưng ``Patient-class'' khi rút trích đặc trưng cho lớp Person.

\begin{table}[th]
\centering\ra{1.2}
\caption{Tập đặc trưng cho lớp Patient \label{tab:PatientFeatures}}
\footnotesize\sffamily

\begin{tabularx}{\textwidth}{@{}P{\raggedright}{0.3}lL@{}}
\toprule 
\textbf{Đặc Trưng} & \textbf{Giá trị} & \textbf{Giải thích}\\
\midrule
Keyword of patient & 0, 1 & Các từ khóa về bệnh nhân (như mr., mr, ms., ms, yo-, y.o., y/o, year-old, ...)\\
Keyword of doctor & 0, 1 & Các từ khóa về bác sĩ (dr, dr., md, m.d., m.d,…)\\
Key word of doctor title & 0, 1 & Các từ khóa về chức vụ của bác sĩ (dentist, orthodontist, …)\\
Key word of department  & 0, 1 & Các từ khóa về chuyên ngành bác sĩ (electrophysiology, …)\\
Key word of general deparment & 0, 1 & Các từ khóa chung về phòng ban (team, service)\\
Key word of general doctor & 0, 1 & Các từ khóa chung về bác sĩ (doctor, dict, author, pcp, attend, provider)\\
Key word of relative & 0, 1 & Các từ khóa về người thân (wife, brother, sibling, nephew)\\
Name & 0, 1 & Có phải là tên riêng hay không\\
Last n line doctor & 0, 1 & Là tên bác sĩ ở n dòng cuối cùng\\
Twin or triplet information & 0, 1 & Thông tin về cặp sinh đôi, sinh ba (baby 1, 2, 3,…)\\
Preceded by non-patient & 0, 1 & Khái niệm đứng trước không phải là bệnh nhân.\\
Signed information  & 0, 1 & Có liên quan đến việc kí/xác nhận bệnh án\\
Previous sentence &  & Câu hoàn chỉnh liền trước khái niệm\\
Next sentence &  & Câu hoàn chỉnh liền sau khái niệm\\
Pronouns we & 0, 1 & Là đại từ chỉ chúng tôi (we, us, our, ourselves)\\
Pronouns I & 0, 1 & Là đại từ chỉ tôi (I, my, me, myself)\\
Pronouns you & 0, 1 & Là đại từ chỉ bạn (you, your, yourself)\\
Pronouns they & 0, 1 & Là đại từ chỉ họ (they, them, their, themselves)\\
Pronouns he/she most & 0, 1 & Thuộc phần đa số của đại từ chỉ cô ấy/anh ấy (he, his, her)\\
Who & 0, 1 & Là đại từ “who” hoặc liền kề với khái niệm đứng trước\\
Appositive & 0, 1 & Là đồng vị ngữ\\
\bottomrule
\end{tabularx}
\end{table}

Các đặc trưng về từ khóa được chúng tôi hiện thực bằng cách khảo sát tập dữ liệu và xây dựng bộ từ điển thích hợp cho từng đặc trưng.

Các đặc trưng ``Previous sentence'' và ``Next sentence'' được hiện thực bằng cách khảo sát toàn bộ các khái niệm thuộc lớp Person, sau đó xây dựng bộ từ điển các câu có thể đứng trước hoặc đứng sau khái niệm đang xét. Giá trị của đặc trưng được lấy bằng chỉ mục của câu đứng trước (hoặc đứng sau) trong bộ từ điển các câu.

Đặc trưng ``Pronouns he/she most'' mang ý nghĩa giới tính chiếm đa số trong BAĐT được xét. Việc xác định giới tính chiếm đa số trong BAĐT được hiện thực bằng cách xác định giới tính cho từng khái niệm thuộc lớp Person, sau đó chọn giới tính có số lượng khái niệm lớn hơn. Phương pháp xác định giới tính được thực hiện theo miêu tả trong đặc trưng của nhóm Person. Nếu trong BAĐT có giới tính Nam chiếm đa số thì những khái niệm là đại từ chỉ về giới tính Nam như ``he'', ``him'', ``himself'', ... sẽ có đặc trưng ``Pronouns he/she most'' mang giá trị là 1. Tương tự cho BAĐT có giới tính Nữ chiếm đa số.

\subsection*{Nhóm Pronoun}

\subsection*{Nhóm Problem/Test/Treatment}

\section{Gom cụm và xây dựng chuỗi đồng tham chiếu}
Ở mô hình cặp thực thể, hệ thống phân loại không có khả năng xây dựng chuỗi đồng tham chiếu mà nó chỉ có thể xác định một cặp khái niệm là có đồng tham chiếu hay không. Mặt khác, đối với một văn bản HSXV, số cặp khái niệm được sinh ra rất nhiều và trong số đó có nhiều cặp có chung khái niệm đứng sau, ví dụ hai cặp “Dr. John”-“his” và “Mr. Brown”-“his” có chung khái niệm đứng sau là “his” mà hai cặp này đều được hệ thống phân loại xác định là đồng tham chiếu, tuy nhiên chỉ một trong hai khái niệm “Dr. John” và “Mr. Brown” được chọn làm tiền đề cho khái niệm “his” này. Như vậy cần thiết phải có một giải thuật lựa chọn các cặp đồng tham chiếu và xây dựng các chuỗi đồng tham chiếu từ chúng.

Như đã được đề cập ở mục, có hai giải thuật được đề xuất là: \emph{gom cụm gần nhất trước} và \emph{gom cụm tốt nhất trước}. Chúng tôi lựa chọn thực hiện giải thuật gom cụm tốt nhất trước cho hệ thống của mình vì hai lý do:
\begin{enumerate}
\item Theo [x], giải thuật gom cụm tốt nhất trước cho kết quả tốt hơn giải thuật gom cụm gần nhất trước.
\item Các tác giả hệ thống [I] cũng hiện thực giải thuật này cho hệ thống của họ.
\end{enumerate}



\section{Đánh giá hiệu năng}

\subsection*{Hệ đo MUC}

\subsection*{Hệ đo B-CUBED}

\subsection*{Hệ đo CEAF}

\chapter{Thí nghiệm đánh giá}
\section{Tập dữ liệu}
Tập dữ liệu chúng tôi sử dụng để đánh giá hiệu năng hệ thống được cung cấp bởi Partners Healthcare và Beth Israel Deaconess Medical Center. Đây cũng chính là tập được sử dụng ở tác vụ 1C của Thách thức i2b2 2011 để đánh giá các hệ thống tham gia. Như vậy để có được tập dữ liệu này, chúng tôi đã liên hệ với trung tâm i2b2 và cam kết về các điều khoản sử dụng dữ liệu (Data Usage Agreement) bao gồm việc chỉ sử dụng cho mục đích nghiên cứu. Bản cam kết cần được kí và gửi lại cho i2b2 qua email hoặc fax.

Tập dữ liệu chúng tôi nhận được bao gồm \emph{251 hồ sơ xuất viện} dùng cho huấn luyện và \emph{173 hồ sơ} dùng cho kiểm tra đánh giá. Trong đó, ngoài các hồ sơ xuất viện là các văn bản thuần được ghi chép lại bởi các bác sĩ, y tá (Hình \ref{hsxv-eg}) thì mỗi hồ sơ còn đi kèm với một tập tin chứa danh sách các khái niệm đã được gán nhãn bởi các chuyên gia y tế theo mẫu: \texttt{c="<khái niệm>" <bắt đầu> <kết thúc>||t="<lớp>"}. Ví dụ \texttt{c="the patient" 20:5 20:6||t="person"} mô tả khái niệm ``the patient'' xuất hiện bắt đầu từ dòng 20 từ thứ 5, kết thúc ở dòng 20 từ thứ 6 và thuộc lớp Person (Hình \ref{con-eg}).

\begin{figure}[ht]
\centering
\lstinputlisting{sample_code/emr_sample.txt}
\caption{Ví dụ nội dung tập tin hồ sơ xuất viện\label{hsxv-eg}}
\end{figure}

\begin{figure}[ht]
\lstinputlisting{sample_code/con_sample.txt}
\caption{Ví dụ nội dung tập tin chứa các khái niệm\label{con-eg}}
\end{figure}

Ngoài danh sách khái niệm, mỗi hồ sơ còn đi kèm với một tập tin chứa danh sách các chuỗi đồng tham chiếu đã được phân giải bởi các chuyên gia (các \emph{chuỗi kết quả}) nhằm huấn luyện hệ thống phân loại có giám sát cũng như để đánh giá hiệu năng của hệ thống phân giải (Hình \ref{chains-eg}).

\begin{figure}[ht]
\lstinputlisting{sample_code/chains_sample.txt}
\caption{Ví dụ nội dung tập tin chứa các chuỗi đồng tham chiếu\label{chains-eg}}
\end{figure}

\section{Các phương pháp đánh giá}
Hiệu năng hệ thống được đánh giá qua ba độ đo: \emph{độ đúng đắn} (precision), \emph{độ đầy đủ} (recall) và \emph{độ F} (F-measure). Bài báo đánh giá các hệ thống dự thi thử thách i2b2 sử dụng ba phương pháp khác nhau để tính toán các độ này, bao gồm: MUC, B-CUBED và CEAF.

\subsection*{Hệ đo MUC}
Hệ đo MUC \cite{MarcVilain1995} xem chuỗi đồng tham chiếu là một danh sách các liên kết giữa các cặp khái niệm tạo nên chuỗi, từ đó đánh giá hệ thống dựa trên số lượng ít nhất các liên kết cần được thêm vào và loại bỏ để các chuỗi đồng tham chiếu xuất ra bởi hệ thống trùng với các chuỗi kết quả. Có thể hiểu số liên kết cần được loại bỏ là độ thiếu chính xác (precision \emph{error}) và số liên kết cần được thêm vào là độ thiếu đầy đủ (recall \emph{error}). 

Với mỗi văn bản $d$, gọi $G$ là tập các chuỗi kết quả, $S$ là tập các chuỗi hệ thống. Các độ đúng đắn ($P$) và độ đầy đủ ($R$) được tính như sau:
\[P_d^{\text{MUC}}=\frac{\sum_{s\in S} \left(|s| - m(s, G)\right)}{\sum_{s\in S}\left(|s| - 1\right)}\]
\[R_d^{\text{MUC}}=\frac{\sum_{g\in G}(|g|-m(g,S))}{\sum_{g\in G}(|g|-1)}\]

\noindent trong đó, $|s|$ là số khái niệm tạo thành chuỗi $s$, $m(s,G)$ được tính là tổng số chuỗi trong $G$ có giao nhau với $s$ cộng với số khái niệm trong $s$ không xuất hiện trong tất cả các chuỗi trong $G$. Độ $F$ của hệ MUC là trung bình điều hòa của độ chính xác và độ đầy đủ:
\[F_d^{\text{MUC}}=\frac{2\times P_d\times R_d}{P_d + R_d}\]

\subsection*{Hệ đo B-CUBED}
Khác với hệ đo MUC đánh giá dựa trên sự thiếu và thừa các liên kết trong chuỗi hệ thống, hệ đo B-CUBED đánh giá hiệu năng trên mỗi khái niệm trong văn bản. Theo nhận định của các tác giả hệ đo B-CUBED, cách đánh giá dựa trên các liên kết có hai nhược điểm \cite{AmitBagga1998}:

\begin{enumerate}[leftmargin=\parindent]
\item Không xét tới các \emph{khái niệm duy nhất} vì các liên kết chỉ tồn tại khi có ít nhất hai khái niệm. Mặt khác, theo quy ước, các chuỗi chỉ chứa một khái niệm không được đưa vào tập kết quả và các hệ thống tuân theo quy ước cũng không xuất những chuỗi như vậy ra.

\item Các lỗi của hệ thống được xem là như nhau, tức hệ đo MUC đánh giá cùng một mức phạt như nhau cho tất cả các sai sót của hệ thống. Tuy nhiên có thể nhận thấy là có vài loại sai sót làm cho chuỗi hệ thống bị sai lệch nhiều hơn so với các loại khác.
\end{enumerate}

Hệ đo B-CUBED được thiết kể để khắc phục hai nhược điểm trên bằng cách tính toán độ chính xác và độ đầy đủ cho từng khái niệm trong văn bản, sau đó kết hợp chúng lại để ra kết quả cuối cùng. Như vậy theo \cite{AmitBagga1998}, với khái niệm thứ $i$, độ chính xác và độ đầy đủ của nó được định nghĩa như sau:
\[P_i=\frac{\text{số khái niệm đúng trong chuỗi hệ thống chứa khái niệm thứ $i$}}{\text{số khái niệm trong chuỗi hệ thống chứa khái niệm thứ $i$}}\]
\[R_i=\frac{\text{số khái niệm đúng trong chuỗi hệ thống chứa khái niệm thứ $i$}}{\text{số khái niệm trong chuỗi kết quả chứa khái niệm thứ $i$}}\]

Độ chính xác và đầy đủ trên toàn văn bản được tính theo công thức:
\[P=\sum_{i=1}^{M} w_i \times P_i,\quad R=\sum_{i=1}^{M} w_i\times R_i\]

\noindent trong đó, $w_i$ là trọng số của khái niệm thứ $i$, $M$ là số khái niệm của văn bản đang xét. Bài báo đánh giá các hệ thống dự thi thử thách i2b2 năm 2011 sử dụng chung một trọng số cho tất cả các khái niệm, $w_i=1/M$. Để tiện cho việc so sánh với hệ thống \cite{YanXu2012}, chúng tôi cũng dùng cách gán trọng số như trên.

Như vậy để thuận tiện trong tính toán, nếu gọi $m$ là một khái niệm trong văn bản $d$ có chứa $M$ khái niệm, $G_m$ là chuỗi kết quả có chứa $m$, $S_m$ là chuỗi hệ thống có chứa $m$, $O_m$ là chuỗi giao nhau giữa $G_m$ và $S_m$ (tức $O_m=G_m\cap S_m$) thì độ đúng đắn và độ đầy đủ của hệ B-CUBED cho $d$ được tính như sau:
\[P_d^{\text{B-CUBED}}=\frac{1}{M}\sum_{m\in d}\frac{|O_m|}{|S_m|},\quad R_d^{\text{B-CUBED}}=\frac{1}{M}\sum_{m\in d}\frac{|O_m|}{|G_m|}\]

Độ F của hệ B-CUBED được tính tương tự như hệ MUC.

\subsection*{Hệ đo CEAF}
Hệ đo CEAF được tác giả Xiaoqiang Lou giới thiệu vào năm 2005 như một cách đánh giá khác để khắc phục các nhược điểm của hệ MUC \cite{XiaoquangLuo2005}. Thay vì tính toán các độ đo trên từng khái niệm như hệ B-CUBED, hệ CEAF đánh giá hệ thống bằng cách tối ưu hóa sự \emph{sắp xếp đầy đủ} giữa các chuỗi kết quả và chuỗi hệ thống dựa trên tổng các độ tương tự của từng cặp chuỗi, với điều kiện một chuỗi kết quả chỉ được ghép cặp với nhiều nhất một chuỗi hệ thống. Theo nhận định của tác giả hệ đo CEAF, sắp xếp tối ưu giúp ngăn ngừa việc ``ăn gian'' của các hệ thống phân giải: một hệ thống xuất ra quá nhiều chuỗi đồng tham chiếu sẽ bị đánh giá độ chính xác thấp, trong khi xuất ra quá ít chuỗi thì sẽ bị đánh giá độ đầy đủ thấp.

Với mỗi văn bản $d$, gọi $G=\{g_i:i=1,2,\dots,|G|\}$ là tập các chuỗi kết quả của $d$, $S=\{s_i:i=1,2,\dots,|S|\}$ là tập các chuỗi hệ thống xuất ra cho $d$. Vì vai trò như nhau của $S$ và $G$ trong hệ đo này, chúng tôi giả định rằng $|S|<|G|$. Theo định nghĩa của \cite{XiaoquangLuo2005}, một sự sắp xếp đầy đủ có thể được xem là một ánh xạ một-một đi từ $S$ vào $G$, $H=\{h:S\mapsto G\}$, thỏa hai điều kiện:
\begin{enumerate}
\item $\forall s \in S,\forall s^{\prime} \in S: s\neq s^{\prime} \Leftrightarrow h(s)\neq h(s^{\prime})$
\item $|H|=|S|$
\end{enumerate}

%\vspace{6pt}
Đặt $m=|S|,\, M=|G|$, $\mathcal{H}$ là toàn bộ những sắp xếp đầy đủ có thể giữa $S$ và $G$, dễ dàng tính được $|\mathcal{H}|=\binom{M}{m}m!$. Gọi $\phi(s,g)$ là độ tương tự giữa hai chuỗi $s$ và $g$ bất kì. Độ tương tự cho một sự sắp xếp đầy đủ $H\in\mathcal{H}$ giữa $S$ và $G$ được định nghĩa là tổng các độ tương tự giữa mỗi cặp $(s,h(s))$ trong $H$: $\Phi(H)=\sum_{s\in S} \phi(s,h(s))$. Như vậy một sự sắp xếp đầy đủ tối ưu giữa $S$ và $G$ chính là sự sắp xếp $H^*$ thỏa:
\begin{align*}
\Phi(H^*)&=\max_{H\in\mathcal{H}} \Phi(H)\\
&=\max_{H\in\mathcal{H}} \sum_{s\in S} \phi(s,h(s))
\end{align*}

Trong trường hợp $|S|>|G|$, vai trò của $S$ và $G$ được hoán đổi cho nhau ở các định nghĩa trên. Để tính độ tương tự giữa hai chuỗi đồng tham chiếu, tác giả hệ CEAF đề xuất bốn cách tính khác nhau:
\begin{align*}
\phi_1(s,g)=
\begin{cases}
	1 & \text{nếu } s = g\\
	0 & \text{nếu } s \neq g
\end{cases}&,\quad
\phi_2(s,g)=
\begin{cases}
	1 & \text{nếu } s\cap g\neq\emptyset\\
	0 & \text{các trường hợp khác}
\end{cases},\\
\phi_3(s,g)=|s\cap g|&,\quad\phi_4(s,g)=\frac{2|s\cap g|}{|s|+|g|}
\end{align*}

Theo nhận định của chính tác giả, $\phi_3$ và $\phi_4$ tỏ ra hiệu quả hơn trong việc đánh giá các chuỗi đồng tham chiếu. Mặt khác, thử thách i2b2 năm 2011 sử dụng $\phi_4$ để đánh giá các hệ thống dự thi nên để cho tiện trong việc so sánh với hệ thống \cite{YanXu2012}, chúng tôi cũng sử dụng $\phi_4$ để đánh giá hệ thống của mình.

Ngoài ra, việc tìm kiếm một sự sắp xếp tối ưu giữa hai tập chuỗi không thể được thực hiện bằng cách duyệt toàn bộ các sự sắp xếp đầy đủ có thể vì số các cách sắp xếp đầy đủ là rất lớn, $\binom{M}{m}m!$. Về mặt giải thuật, cách làm này được xem là có độ phức tạp hàm giai thừa. Tuy nhiên theo nhận định của tác giả CEAF, bài toán sắp xếp tối ưu này chính là bài toán giao công việc (\emph{assignment problem}) hay bài toán vận chuyển (\emph{transportation problem}) đã được giải quyết bởi Harold W. Kuhn vào năm 1955 với giải thuật có tên là \emph{phương pháp Hungary} có độ phức tạp là $\BigO(n^4)$ \cite{HungarianMethod}.

Từ đây, hệ CEAF tính độ đúng đắn và độ đầy đủ cho một văn bản $d$ theo công thức:
\[P_d^{\text{CEAF}}=\frac{\Phi(H^*)}{\sum_{s\in S} \phi(s,s)},\quad R_d^{\text{CEAF}}=\frac{\Phi(H^*)}{\sum_{g\in G} \phi(g,g)}\]

Độ F của hệ CEAF được tính tương tự như hệ MUC.

\section{Kết quả}
Vì mỗi hệ đo nêu trên đánh giá các chuỗi đồng tham chiếu cho một văn bản HSXV duy nhất nên ứng với mỗi hệ đo, trung bình không trọng số của mỗi độ đo tính trên toàn tập kiểm tra được lấy làm kết quả của hệ đo đó. Gọi $D$ là tập gồm $N$ HSXV dùng để kiểm tra hệ thống, kết quả đo trung bình của hệ MUC được tính như sau:
\[P^{\text{MUC}}=\frac{\sum_{d\in D}P_d^{\text{MUC}}}{N},\quad R^{\text{MUC}}=\frac{\sum_{d\in D}R_d^{\text{MUC}}}{N},\quad F^{\text{MUC}}=\frac{\sum_{d\in D}F_d^{\text{MUC}}}{N}\]

Các hệ B-CUBED và CEAF được tính tương tự. Cuối cùng, trung bình không trọng số của các hệ được lấy làm kết quả chung cuộc để đánh giá hệ thống:
\[P=\frac{P^{\text{MUC}}+P^{\text{B-CUBED}}+P^{\text{CEAF}}}{3}\]
\[R=\frac{R^{\text{MUC}}+R^{\text{B-CUBED}}+R^{\text{CEAF}}}{3}\]
\[F=\frac{F^{\text{MUC}}+F^{\text{B-CUBED}}+F^{\text{CEAF}}}{3}\]

Các kết quả đánh giá hệ thống của chúng tôi với đầy đủ các đặc trưng được trình bày trên Bảng \ref{final-result} với kết quả chung cuộc là: độ chính xác 0,847, độ đầy đủ 0,795 và độ F là 0,815. Kết quả phân giải cho các chuỗi thuộc lớp Person rất khả quan với độ F là 0.890. Điều này cho thấy sự khác biệt giữa miền văn bản BAĐT với các miền văn bản khác, trong đó thông tin về bệnh nhân đóng góp một phần không nhỏ.

\begin{rtable}{Kết quả đánh giá hệ thống\label{final-result}}
Tất cả & 0,882 & 0,817 & 0,845 && 0,921 & 0,873 & 0,895 && 0,737 & 0,695 & 0,707 && \underline{0,847} & \underline{0,795} & \underline{0,815} \\
Person & 0,965 & 0,928 & 0,944 && 0,939 & 0,895 & 0,910 && 0,872 & 0,795 & 0,816 && 0,925 & 0,873 & 0,890 \\
Problem & 0,742 & 0,633 & 0,666 && 0,814 & 0,674 & 0,689 && 0,685 & 0,653 & 0,654 && 0,747 & 0,653 & 0,670 \\
Test & 0,631 & 0,558 & 0,568 && 0,812 & 0,676 & 0,637 && 0,634 & 0,580 & 0,589 && 0,692 & 0,604 & 0,598 \\
Treatment & 0,795 & 0,763 & 0,759 && 0,821 & 0,805 & 0,772 && 0,779 & 0,772 & 0,761 && 0,798 & 0,780 & 0,764 \\
\end{rtable} 

Thực nghiệm của chúng tôi cho thấy hiệu năng của mô hình phân loại bệnh nhân dựa trên kiểm chứng chéo 5 mẫu (5-fold cross validation) có độ chính xác, độ đầy đủ và độ F lần lượt là 0,948; 0,961 và 0,955. Bảng \ref{tab:person-cmp} so sánh kết quả đánh giá phân giải đồng tham chiếu cho lớp Person với việc sử dụng và không sử dụng đặc trưng Patient-class. Hệ thống với đặc trưng Patient-class có hiệu năng được cải thiện lên đến 21,75\%, điều này cho thấy thông tin về bệnh nhân trong miền văn bản BAĐT là một thông tin rất hữu ích.

Ngoài ra, các kết quả đánh giá cho các lớp Problem, Test và Treatment của chúng tôi có độ F lần lượt là 0,670; 0,598 và 0,764. Tổng kết lại, chúng tôi rút ra một số nhận xét và giải thích về các kết quả đánh giá này:
\begin{enumerate}
\item Kết quả của lớp Test là thấp nhất vì theo quan sát của chúng tôi, tập vector đặc trưng dùng để huấn luyện của lớp Test chịu ảnh hưởng nhiều nhất bởi sự mất cân bằng mẫu (khi chưa qua chọn lọc mẫu, tỉ lệ mẫu âm trên mẫu dương là 381:1). Mặt khác, nhiều thủ tục y tế trong một hồ sơ xuất viện trùng tên hoàn toàn hoặc một phần với nhau nhưng lại là những khái niệm duy nhất không đồng tham chiếu với bất kì khái niệm nào, điều này khiến cho các đặc trưng so trùng chuỗi và tri thức nhân loại bị ảnh hưởng.
\item Các kết quả đánh giá hệ thống của chúng tôi đều thấp hơn so với hệ thống \cite{YanXu2012}. Điều này có thể được giải thích một cách chủ quan như sau: các nhược điểm của mô hình cặp khái niệm khiến cho hiệu năng của mô hình này phụ thuộc rất nhiều vào hiệu năng của các mô hình phân loại. Vì giới hạn về thời gian và công nghệ, nhiều mô hình SVM được chúng tôi huấn luyện với bộ tham số mặc định mà không thông qua tìm kiếm lưới.
\item Mặt khác, tuy chúng tôi đã hiện thực lại phần lớn các đặc trưng được nêu trong hệ thống \cite{YanXu2012}, các sai sót về chi tiết kĩ thuật là không thể tránh khỏi, ví dụ như sự khác biệt giữa các tập từ khóa và các biểu thức chính quy, hay sự khác nhau trong thuật toán so sánh trùng chuỗi và tìm kiếm từ điển. Bênh cạnh đó còn có một số đặc trưng không được chúng tôi hiện thực, ví dụ đặc trưng giả định của thuốc (assertion), đặc trưng về không gian (spatial) hay đặc trưng so trùng dữ liệu MSRA sử dụng Probase--một nguồn tri thức nhân loại được xây dựng nội bộ bởi Micrsoft Research Asia (nơi các tác giả hệ thống \cite{YanXu2012} làm việc) mà chúng tôi không thể tiếp cận được.
\end{enumerate}

\begin{table}[th]
\centering\ra{1.3}
\caption{Hiệu quả phân giải lớp Person khi có đặc trưng Patient-class\label{tab:person-cmp}}
\footnotesize\sffamily
\begin{tabularx}{0.9\textwidth}{@{}lCCCC@{}}
\toprule
& \textbf{P} & \textbf{R} & \textbf{F} & \% tăng lên\\
\midrule
Không có đặc trưng Patient-class & 0,770 & 0,755 & 0,731 &\\
Có đặc trưng Patient-class & 0,925 & 0,873 & 0,890 & 21,75\%\\
\bottomrule
\end{tabularx}
\end{table}

\chapter{Tổng kết}

\section{Kết quả đạt được}
Sau quá trình nghiên cứu và tìm hiểu các phương pháp phân giải đồng tham chiếu cho các văn bản nói chung và bệnh án điện tử nói riêng, chúng tôi đã hoàn thành hệ thống phân giải đồng tham chiếu cho các HSXV tiếng Anh với độ F là 81.5\%. Các kiến thức chúng tôi thu thập được trong quá trình hiện thực luận án có thể làm nền tảng cho những phát triển sâu hơn của việc khai thác tri thức trong bệnh án điện tử trong tương lai, cụ thể là:
\begin{enumerate}
\item Đối với các văn bản nói chung, có ba mô hình được đề xuất để giải quyết bài toán phân giải đồng tham chiếu: mô hình cặp khái niệm, mô hình đề cập thực thể và mô hình xếp hạng. Mô hình cặp khái niệm có tư tưởng đơn giản nhất nhưng vì thế nó cũng có nhiều nhược điểm, vì vậy một số vấn đề cần phải được cân nhắc trong quá trình hiện thực như sự mất cân bằng lớp, giải thuật học máy, các đặc trưng được rút trích và các phương pháp gom cụm. Mô hình đề cập thực thể được đề xuất nhằm giải quyết vấn đề về sự độc lập của việc xác định tính đồng tham chiếu của các cặp khái niệm, bằng cách đưa vào các đặc trưng ở mức cụm. Trong khi đó mô hình xếp hạng lại nhắm tới vấn đề lựa chọn tiền đề, bằng cách xếp hạng các ứng cử viên để chọn ra tiền đề phù hợp nhất. 
\item Về vấn đề mất cân bằng lớp, có hai cách khác nhau để giải quyết việc này: chọn lọc mẫu (undersampling) hoặc tạo thêm mẫu (oversampling). Việc tạo thêm mẫu giúp cho các mẫu dương ngang bằng với các mẫu âm làm cho số lượng mẫu tăng lên, trong một bài toán có nhiều mẫu được sinh ra thì cách làm này khiến cho thời gian huấn luyện mô hình phân loại tăng lên rất nhiều, nhất là đối với SVM. Trong khi đó việc chọn lọc mẫu một cách tự động lại khiến cho nhiều mẫu chứa các thông tin quan trọng bị loại bỏ, dẫn tới việc mô hình phân loại có xu hướng phân loại về dương nhiều hơn mà bỏ quá sự chính xác của chúng.
\item Đối với bệnh án điện tử, nhiều đặc điểm chuyên biệt trong miền văn bản này giúp cho việc phân giải đồng tham chiếu được dễ dàng. Điển hình là thông tin về sự đề cập đến bệnh nhân, sự giống nhau về chuỗi kí tự của các khái niệm đồng tham chiếu chỉ về bệnh, thủ tục y tế và phương pháp điều trị hay các thông tin về ngữ cảnh y tế trong bệnh án. Ngoài ra đặc điểm một bệnh án chỉ đề cập tới một bệnh nhân cũng là nhân tố quan trọng trong việc phân giải đồng tham chiếu cho các khái niệm chỉ người ở miền văn bản này.
\end{enumerate}

Về mặt thực tiễn, tuy hệ thống của chúng tôi không được hiện thực cho các văn bản tiếng Việt, chúng tôi hy vọng rằng những kết quả đạt được ở luận án này sẽ đóng góp tích cực vào việc nghiên cứu và phát triển BAĐT ở Việt Nam.

\section{Hạn chế và hướng phát triển}
\subsection*{Hạn chế}
Ngoài những kết quả đạt được, hệ thống của chúng tôi vẫn còn một số hạn chế nhất định. Thứ nhất là thời gian chạy thử nghiệm tốn nhiều thời gian. Hạn chế này dẫn đến việc chúng tôi không có đủ thời gian để điều chỉnh hệ thống và thực hiện các thí nghiệm nhiều lần. Để cải thiện vấn đề này, chúng tôi đề xuất sử dụng các máy tính có khả năng tính toán mạnh trong quá trình huấn luyện hoặc áp dụng các kĩ thuật xử lý song song và hệ phân bố cho việc tìm kiếm lưới.

Thứ hai là hạn chế trong việc trích xuất các thông tin từ điển UMLS, Wikipedia và các thông tin ngữ cảnh cho các cặp khái niệm lớp Problem, Test và Treatment. Để rút ngắn thời gian huấn luyện cũng như chạy thử nghiệm trên tập kiểm tra, chúng tôi đã tiến hành rút trích các đặc trưng được nêu thành các tập tin được lưu trữ và sử dụng lại nhiều lần. Tuy nhiên cách hiện thực này làm giảm sự linh động của quá trình phân giải vì đối với các HSXV mới không có sẵn các tập tin này, chúng cần được trích xuất trước khi tiến hành phân giải.

\subsection*{Hướng phát triển}
Vì giới hạn thời gian hiện thực luận án cùng với những hạn chế nêu trên, nhiều ý tưởng và hướng phát triển không được chúng tôi hiện thực. Một trong số đó là phát triển để hệ thống hoạt động được với HSXV bằng tiếng Việt. Để làm được điều này chúng tôi cần thay thế các công cụ xử lý ngôn ngữ tự nhiên từ tiếng Anh qua tiếng Việt, đồng thời xây dựng lại các bộ từ điển cho các đặc trưng. Thông qua tìm hiểu, chúng tôi biết được hiện nay đã có một số công cụ hỗ trợ xử lý ngôn ngữ tự nhiên cho tiếng Việt như: công cụ JVnTextPro và Framework VNLP. Vì vậy chúng tôi tin rằng nếu tìm được tập dữ liệu tiếng Việt phù hợp, việc phát triển để hệ thống hoạt động được cho HSXV tiếng Việt là hoàn toàn khả thi.

Ngoài ra, chúng tôi cũng dự định kết hợp hệ thống hiện tại với các hệ thống nhận diện thực thể để tạo thành một công cụ hoàn chỉnh giúp phân giải cho các văn bản BAĐT thô mà không cần danh sách các thực thể cho trước. Hiện nay đã tồn tại nhiều công cụ giúp thực hiện bước nhận diện thực thể như: \textit{CliNER} của nhóm Text Machine Lab cho các văn bản tiếng Anh hoặc kết quả từ luận văn \textit{Nhận dạng tên thực thể trong bệnh án điện tử}, được thực hiện bởi Đào Trọng Điệp và Lê Khắc Sinh, cho các văn bản tiếng Việt. Việc kết hợp này yêu cầu thời gian tìm hiểu mã nguồn và cách sử dụng các công cụ nêu trên.


\clearpage
\phantomsection
\addcontentsline{toc}{chapter}{\bibname}
\bibliographystyle{ieeetr}
\bibliography{referrence}

\end{document}